\documentclass[12pt, letterpaper]{article}
\usepackage[utf8]{inputenc}
\usepackage{listings}
\usepackage{color}
\usepackage{soul}

\definecolor{dkgreen}{rgb}{0,0.6,0}
\definecolor{gray}{rgb}{0.5,0.5,0.5}
\definecolor{mauve}{rgb}{0.58,0,0.82}

\lstset{frame=tb,
  language=Python,
  aboveskip=3mm,
  belowskip=3mm,
  showstringspaces=false,
  columns=flexible,
  basicstyle={\small\ttfamily},
  numbers=none,
  numberstyle=\tiny\color{gray},
  keywordstyle=\color{blue},
  commentstyle=\color{dkgreen},
  stringstyle=\color{mauve},
  breaklines=true,
  breakatwhitespace=true,
  tabsize=3
}

\setcounter{secnumdepth}{0}

\title{PROVA 1 - SEMESTRE 2021.2} 
\author{Pedro Santi Binotto [20200634]}
\date{\today}

\begin{document}
\maketitle

% QUESTAO 1
\newpage
\section{Questão 1}
\subsection{ITEM A)}
\paragraph{}
Utilizando uma lista encadeada simples, é possível manter um registro de cada palavra
(e quantas vezes a mesma foi utilizada) na forma de um "nodo" da lista, realizando o
encadeamento direto de cada palavra diferente na ordem em que esta aparece no texto.

\paragraph{}
Para ilustrar esta solução, foram escritas (em Python \st{eca}) as classes referentes à
lista encadeada, os nodos, e uma classe "Reader", com o propósito de encapsular o a
lógica utilizada ao ler o texto que servirá de exemplo na demonstração.

(Todo o código fonte apresentado neste documento (e mais!) está disponível
no diretório "fonte" do projeto)
\subsubsection{Nodo}
\paragraph{}
A classe "Node" apresenta campos para o texto do elemento, isto é, a palavra em sí
- a quantidade de vezes que foi utilizada, e um ponteiro para o próximo elemento na lista:
\begin{lstlisting}
# Classe "Nodo"
class Node:
    def __init__(self, text: str, next=None):
        self.__quant = 1
        self.__text = text
        self.__next = next

    # getters e setters ... (disponiveis no fonte completo)
\end{lstlisting}

\subsubsection{Lista}
\paragraph{}
A classe "Lista" permite adicionar elementos e procurar por palavras com base no conteúdo
do texto. A busca é realizada com base na iteração sobre cada elemento até que se econtre
a palavra desejada ou que o fim da lista seja alcançado. Na adição, realiza-se uma busca
sobre os elementos pela palavra; caso esta já esteja presente na lista, incrementa-se o
valor do campo de contagem do elemento - caso contrário, adiciona-se um novo elemento ao
fim da lista.
\begin{lstlisting}
# Classe "Lista"
class LinkedList:
    def __init__(self):
        self.__head: Node = None
        self.__tail: Node = None

    def __addElem(self, elem: Node):
        if self.__head == None:
            self.__head = elem
            self.__tail = elem
            return
        self.__tail.next = elem
        self.__tail = elem

    def storeWord(self, word: str):
        word = word.lower().translate(str.maketrans('', '', string.punctuation))
        tmp = self.__head
        if tmp == None:
            self.__addElem(Node(word))
            return
        while tmp.next is not None:
            if tmp.text == word:
                tmp.quant += 1
                return
            tmp = tmp.next
        self.__addElem(Node(word))

    # implementacao de demais metodos disponiveis no fonte completo
\end{lstlisting}

\subsubsection{Reader}
\paragraph{}
A classe "Reader" serve o propósito de abstrair o funcionamento necessário para ler o
texto base, irrelevante para os propósitos desta demonstração.
\begin{lstlisting}
# Classe "Reader"
class Reader:
    def __init__(self, textFile):
        text = textFile.open()
        self.__words = self.__parseFile(text)
        self.__cursorPos = 0
        text.close()

    # implementacao e metodos privados disponiveis no fonte completo

    def readNextWord(self):
        # [...]
\end{lstlisting}

\paragraph{}
Fazendo uso destas classes, é possível adicionar palavras com base em um texto de exemplo
(neste caso, o texto está disponível em ./fonte/questao1/textoExemplo.txt) e, após isso,
visualizar todas as palavras usadas e quantas vezes cada uma foi repetida.

\begin{lstlisting}
def itemA():
    scriptLocation = Path(__file__).absolute().parent
    fileLocation = scriptLocation / 'textoExemplo.txt'

    lista = LinkedList()
    r = Reader(fileLocation)
    while True:
        el = r.readNextWord()
        if el == None:
            break
        lista.storeWord(el)

    tmp = lista.head
    while True:
        print( f"""{{ PALAVRA: '{tmp.text}', QUANTIDADE: {tmp.quant} }},""")
        if tmp.next == None:
            return
        tmp = tmp.next
\end{lstlisting}

Saída:

\begin{lstlisting}
{ PALAVRA: 'spam', QUANTIDADE: 15 },
{ PALAVRA: 'lovely', QUANTIDADE: 2 },
{ PALAVRA: 'wonderful', QUANTIDADE: 2 },
{ PALAVRA: 'yes', QUANTIDADE: 1 },
{ PALAVRA: 'monty', QUANTIDADE: 1 },
{ PALAVRA: 'python', QUANTIDADE: 1 },
{ PALAVRA: 'unwittingly', QUANTIDADE: 1 },
{ PALAVRA: 'inspired', QUANTIDADE: 1 },
{ PALAVRA: 'the', QUANTIDADE: 3 },
{ PALAVRA: 'current', QUANTIDADE: 1 },
[ ... ]
\end{lstlisting}

\subsection{ITEM B)}
\paragraph{}
Utilizar uma tabela hash para realizar a mesma tarefa do item anterior produz uma solução
mais eficiente quando se analisa o tempo de busca de um elemento qualquer em meio aos
demais. Ao indexar os elementos de acordo com o seu conteúdo (atravéz de uma função hash),
a probabilidade de colisões entre os elementos torna-se menor conforme a tabela cresce
em capacidade (contanto que utilize-se uma função que gere resultados diversos o
suficiente), assim aproximando o tempo de acesso dos elementos à O(1) (em casos ideais).

\paragraph{}
Para demonstrar a diferença em eficiência, foi desenvolvida uma classe "HashTable",
também em Python, que armazena dados do mesmo tipo de "Nodo" que a classe "LinkedList"
previamente observada:

\begin{lstlisting}
class HashTable:
    def __init__(self, tableSize: int):
        self.__size = tableSize
        self.__table = [LinkedList()] * self.__size

    @staticmethod
    def __hash(val: str, max: int) -> int:
        return sum([ord(char) for char in val]) \% max

    def storeWord(self, word: str):
        self.__table[self.__hash(word, self.__size)].storeWord(word)

    def lookUpWord(self, word: str):
        return self.__table[self.__hash(word, self.__size)].lookUpWord(word)
\end{lstlisting}

\paragraph{}
Após isso, a performance de cada estrutura foi testada e temporizada contra o mesmo
texto-base (neste caso, o roteiro do filme "O Grande Lebowski"); armazenando
todas as palavras do texto e realizando operações de busca pelas mesmas palavras
em cada caso.

\begin{lstlisting}
# item_a.py
def itemABenchmark():
    scriptLocation = Path(__file__).absolute().parent
    fileLocation = scriptLocation / 'biglebowski.txt'

    lista = LinkedList()
    r = Reader(fileLocation)
    while True:
        el = r.readNextWord()
        if el == None:
            break
        lista.storeWord(el)

    lista.lookUpWord('dude')
    lista.lookUpWord('opinion')
    lista.lookUpWord('dollars')
    lista.lookUpWord('vietnam')

# item_b.py
def itemB():
    scriptLocation = Path(__file__).absolute().parent
    fileLocation = scriptLocation / 'biglebowski.txt'

    tabela = HashTable(10)
    r = Reader(fileLocation)
    while True:
        el = r.readNextWord()
        if el == None:
            break
        tabela.storeWord(el)

    tabela.lookUpWord('dude')
    tabela.lookUpWord('opinion')
    tabela.lookUpWord('dollars')
    tabela.lookUpWord('vietnam')
\end{lstlisting}

Resultados:

\begin{lstlisting}
# utilizando uma tabela de tamanho 10
$ time python3 item_a.py
python3 item_a.py  5,88s user 0,04s system 99\% cpu 5,956 total

$ time python3 item_b.py
python3 item_b.py  5,28s user 0,01s system 99\% cpu 5,315 total

$ time python3 item_a.py
python3 item_a.py  6,14s user 0,01s system 99\% cpu 6,175 total

$ time python3 item_b.py
python3 item_b.py  5,73s user 0,01s system 99\% cpu 5,760 total

# lookUpWord('dude')    => 786 repeticoes
# lookUpWord('opinion') =>   1 repeticoes
# lookUpWord('dollars') =>  11 repeticoes
# lookUpWord('vietnam') =>   4 repeticoes
\end{lstlisting}

\subsection{ITEM C)}
\paragraph{}
Para encontrar a palavra mais utilizada, ou ordenar as palavras de qualquer maneira,
o hash não apresenta nenhuma vantagem significativa sobre a lista encadeada - o hashing,
afinal, é feito sobre o conteúdo (texto) de cada palavra, portanto só torna-se mais
eficiente em relação à lista quando o acesso é feito com base neste mesmo critério.
Ao buscar pelo elemento de maior número (de repetições), seria necessário iterar sobre os
elementos "cegamente", como no caso da lista encadeada.

\subsection{ITEM D)}
% TODO

% QUESTAO 2
\newpage
\section{Questão 2}
\subsection{ITEM A)}
\paragraph{}
De forma geral, quando se analisa a performance de algoritmos e estruturas de dados, os
dois grandes fatores à serem considerados são a complexidade de tempo e de espaço, isto é,
 quanto tempo é consumido para executar dada operação (geralmente relativo ao tamanho do
 input, ou quantidade de dados), e a memória que será utilizada para executar estas
 operações.

\paragraph{}
Estes fatores, em combinação, são o que geralmente se denomina "complexidade computacional",
e como já disse Terry Davis, "Um bobão admira a complexidade,
um gênio admira a simplicidade".

\subsection{ITEM B)}
\paragraph{}
Para representar os diferentes possíveis estados das soluções apresentadas, será utilizada
uma tabela, que denota (em "Big O") a complexidade computacional de cada cenário para
cada solução.

\begin{center}
\begin{tabular}{ | c | c | c | c | c | }
 \hline
    & INSERCAO [CM] & BUSCA [CM] & INSERCAO [PC] & BUSCA [PC] \\
 \hline\hline
    ARRAY & O(1) & O(1) & O(n) & O(n) \\
    LISTA & O(1) & O(n) & O(n) & O(n) \\
\hline
\end{tabular}
\end{center}

\paragraph{}
Como pode ser observado, a complexidade espacial é equivalente entre as duas soluções,
isto ocorre pois em ambos os casos a memória alocada é (praticamente) dada apenas pelos
nodos gerados na execução das tarefas, que depende apenas do tamanho de input de dados.

\paragraph{}
Na complexidade temporal, no entanto, pode ser claramente observada a vantagem que a tabela
construida com o array apresenta; ao indexar os elementos à partir do hashing, aproximamos
o tempo de execução das operações ao tempo constante (O(1)). Isto ocorre por que, quando
não ocorrem conflitos, a posição do elemento na lista já é conhecida antes mesmo de
acessa-la. Em comparação, utilizando a lista encadeada, a performance observada é
equivalente ao caso de conflito de quando se utiliza o array: é necessário iterar sobre
os elementos até que se encontre o valor desejado, aproximando-nos à O(n).

% QUESTAO 3
\newpage
\section{Questão 3}
\subsection{ITEM A)}
\paragraph{}
Sim, afetaria. Partindo do pressuposto que o fator de carga (C) é dado pela proporção
matemática entre a quantidade de chaves (itens, N) e a quantidade de slots (M) no
(C = N/M) array da tabela, ao diminuir o número de itens na tabela, diminui-se o fator
de carga da tabela também.

\subsection{ITEM B)}
\paragraph{}
Tratando de HashTables tradicionais, isto é, que utilizam encadeamento direto em colisões,
cada elemento no array de espelho pode representar a "cabeça" de uma lista encadeada - que,
por natureza, permite reconfigurar seu elementos sem precisar realocar os itens
não-alterados. Isto se dá por que a lista encadeada é alocada um item por vez;
em contraste, um array tradicional (estilo C) é alocado "todo de uma vez", em um bloco
contínuo na memória.

\paragraph{}
Sendo assim, ao remover os itens que não atendem os critérios para permanecer, tudo o
que ocorre é que se "re-aponta" os ponteiros dos nodos vizinhos para acomodar a nova
configuração dos dados.

\paragraph{}
Portanto, conclui-se que os  elementos que não atenderem o critério podem simplesmente
ser deixados no lugar (contanto que a transferência dos demais tenha sido realizada
competentemente), bastando apenas renomear "Monstrinho".

\subsection{ITEM C)}
\paragraph{}
O intervalo possível de fator de carga para "Filhote" estende-se de 0 (caso elemento
nenhum satisfaça os critérios para ser incluso) à N, sendo N a carga original de
monstrinho (considerando que Filhote tenha sido instanciado com o mesmo número de
"espaços" disponiveis no array de espelho que Monstrinho), no caso de todos os elementos
aderirem aos critérios.

\paragraph{}
Em relação à "Pequenino", a tendência é do fator de carga diminuir ao ritmo que os itens
originais presentes em "Monstrinho" satisfazem (ou não) os requisitos para serem
transferidos para a tabela "Filhote".

\paragraph{}
Sendo assim, o novo fator de carga de "Pequenino"
é determinado por "C = (N - X) / M", onde C representa o fator de carga; M, o número de
espaços; N, o número original de itens (de "Monstrinho"); e X, o número de itens que
atendem os critérios de transferência, que pode variar entre 0 e N.

\subsection{ITEM D)}
\paragraph{}

% QUESTAO 4
\newpage
\section{Questão 4}
PROVA PROVA PROVA PROVA PROVA PROVA PROVA

% QUESTAO 5
\newpage
\section{Questão 5}
PROVA PROVA PROVA PROVA PROVA PROVA PROVA
\end{document}

